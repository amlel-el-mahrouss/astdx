\documentclass{article}
\usepackage[a4paper,margin=1in]{geometry}
\usepackage{listings}
\usepackage{xcolor}
\usepackage{amsmath}
\usepackage{hyperref}
\usepackage{longtable}
\usepackage{graphicx}
\usepackage{fancyhdr}
\usepackage{titlesec}
\usepackage{caption}

\titleformat{\section}{\normalfont\Large\bfseries}{\thesection}{1em}{}

\definecolor{codegray}{gray}{0.95}
\lstset{
  backgroundcolor=\color{codegray},
  basicstyle=\ttfamily\small,
  breaklines=true,
  frame=single,
  tabsize=4,
  language=C++,
  showstringspaces=false
}

\title{Embedded File System (EMBFS) Specification}
\author{Amlal El Mahrouss}
\date{2025}

\begin{document}

\maketitle

\section{Overview}
The \textbf{Embedded File System (EMBFS)} is a compact and minimal filesystem designed for use in embedded environments with limited storage media. This specification describes its key data structures, constants, and layout.

\section{Namespace}
The EMBFS implementation resides under:
\begin{lstlisting}
namespace astdx::embdfs
\end{lstlisting}

\section{Supported Logical Block Addressing (LBA) Modes}
Two LBA addressing modes are supported via macro definitions:
\begin{itemize}
    \item \texttt{EMBDFS\_28BIT\_LBA} — Uses \texttt{uint32\_t}.
    \item \texttt{EMBDFS\_48BIT\_LBA} — Uses \texttt{uint64\_t}.
\end{itemize}

\section{Fundamental Types}
\begin{longtable}{|l|l|}
\hline
\textbf{Type} & \textbf{Definition} \\
\hline
\texttt{lba\_t} & 28-bit or 48-bit logical block address \\
\texttt{sword\_t} & \texttt{int16\_t} \\
\texttt{sdword\_t} & \texttt{int32\_t} \\
\texttt{utf8\_char\_t} & \texttt{uint8\_t}, used for file names and labels \\
\hline
\end{longtable}

\section{Constants}
\begin{longtable}{|l|l|}
\hline
\textbf{Constant} & \textbf{Description} \\
\hline
\texttt{\_superblock\_name\_len} & Length of superblock name (16 bytes) \\
\texttt{\_superblock\_reserve\_len} & Reserved bytes in superblock (462 bytes) \\
\texttt{\_inode\_name\_len} & Length of inode file name (128 bytes) \\
\texttt{\_inode\_arr\_len} & Number of inodes per partition (12) \\
\texttt{\_inode\_lookup\_len} & Number of direct LBA pointers in inode (8) \\
\hline
\end{longtable}

\section{Superblock Structure}
\textbf{Structure:} \texttt{embdfs\_superblock}
\begin{lstlisting}
struct embdfs_superblock {
    sword_t     s_block_mag;
    sdword_t    s_num_inodes;
    sdword_t    s_part_size;
    sdword_t    s_part_used;
    sdword_t    s_version;
    sword_t     s_sector_sz;
    lba_t       s_inode_start;
    lba_t       s_inode_end;
    utf8_char_t s_name[16];
    utf8_char_t s_reserved[462];
};
\end{lstlisting}

\textbf{Description:}
\begin{itemize}
    \item \texttt{s\_block\_mag} — Magic number identifying the filesystem
    \item \texttt{s\_num\_inodes} — Total number of inodes
    \item \texttt{s\_part\_size} — Total size of the partition (in sectors)
    \item \texttt{s\_part\_used} — Used size of the partition
    \item \texttt{s\_version} — Filesystem version
    \item \texttt{s\_sector\_sz} — Sector size (in bytes)
    \item \texttt{s\_inode\_start}, \texttt{s\_inode\_end} — LBA range where inodes are stored
    \item \texttt{s\_name} — Filesystem label
    \item \texttt{s\_reserved} — Reserved for future use
\end{itemize}

\section{Inode Structure}
\textbf{Structure:} \texttt{embdfs\_inode}
\begin{lstlisting}
struct embdfs_inode {
    utf8_char_t i_name[128];
    sword_t     i_size_virt, i_size_phys;
    lba_t       i_offset[8];
    sword_t     i_checksum, i_flags_perms;
    lba_t       i_acl_creat, i_acl_edit, i_acl_delet;
};
\end{lstlisting}

\textbf{Description:}
\begin{itemize}
    \item \texttt{i\_name} — UTF-8 encoded file name
    \item \texttt{i\_size\_virt} — Virtual file size (logical size)
    \item \texttt{i\_size\_phys} — Physical size in sectors
    \item \texttt{i\_offset} — Array of direct LBA pointers
    \item \texttt{i\_checksum} — CRC32 checksum of file contents
    \item \texttt{i\_flags\_perms} — Flags and permissions field
    \item \texttt{i\_acl\_creat}, \texttt{i\_acl\_edit}, \texttt{i\_acl\_delet} — ACLs for access control
\end{itemize}

\section{Inode Array Type}
\begin{lstlisting}
typedef embdfs_inode embdfs_inode_arr_t[12];
\end{lstlisting}
Represents a fixed-size array of 12 inodes within a given partition.

\section{License}
\begin{itemize}
    \item Copyright © 2025 Amlal El Mahrouss
    \item All rights reserved.
    \item For questions or licensing requests, contact: \texttt{amlal@nekernel.org}
\end{itemize}

\end{document}